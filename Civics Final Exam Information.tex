\documentclass[12pt, a4paper, oneside]{ctexbook}
% 全角符号转换为半角符号
\usepackage{newunicodechar}
\newunicodechar{。}{.}
\newunicodechar{,}{,}
\newunicodechar{:}{:}
\newunicodechar{;}{;}
\newunicodechar{!}{!}
\newunicodechar{(}{(}
\newunicodechar{)}{)}
\usepackage{amsmath, amsthm, amssymb, bm, graphicx, enumitem, hyperref, mathrsfs}

\title{{\Huge{\textbf{思政期末大题汇总}}}}
\author{邹文杰}
\date{\today}
\linespread{1.5}
\newtheorem{theorem}{定理}[section]
\newtheorem{definition}[theorem]{定义}
\newtheorem{lemma}[theorem]{引理}
\newtheorem{corollary}[theorem]{推论}
\newtheorem{example}[theorem]{例}
\newtheorem{proposition}[theorem]{命题}

\begin{document}

\maketitle

\pagenumbering{roman}
\setcounter{page}{1}

% \begin{center}
%     \Huge\textbf{前言}
% \end{center}~\


% ~\\
% \begin{flushright}
%     \begin{tabular}{c}
%         邹文杰\\
%         \today
%     \end{tabular}
% \end{flushright}

% \newpage
\pagenumbering{Roman}
\setcounter{page}{1}
\tableofcontents
\newpage
\setcounter{page}{1}
\pagenumbering{arabic}

\chapter{毛概期末大题}

\section{马克思主义中国化时代化的必要性(重要)}

马克思主义中国化时代化,就是立足中国国情和时代特点,坚持把马克思主义基本原理同中国具体实际相结合、同中华优秀传统文化系相结合。

\begin{enumerate}[label=(\arabic*)]
\item 第一个原因:推进马克思主义中国化时代化,是马克思主义理论本身发展的内在要求。作为洞察使得、引领时代的科学理论,马克思主义只有正确运用于实践并在实践中不断发展才能体现其科学性,彰显其强大力量。马克思主义只有实现中国化时代化,才能不断发展自身,始终保持蓬勃生机和旺盛活力。
\item 第二个原因:推进马克思主义中国化时代化,是解决中国实际问题的客观需要。马克思主义要在中国发挥指导作用,就必须中国化时代化。只有与中国国情相结合,与时代发展同进步,马克思主义才能真正解决中国的实际问题。
\end{enumerate}

\section{马克思主义中国化时代化的科学内涵}

\begin{enumerate}[label=(\arabic*)]
\item 运用马克思主义的立场、观点和方法,观察时代、把握时代、引领时代,解决中国革命、建设、改革中的实际问题。

\item 总结和提炼中国革命、建设、改革的实践经验并将其上升为理论,不断丰富和发展马克思主义的理论宝库,赋予马克思主义以新的时代内涵。

\item 运用中国人民喜闻乐见的民族语言来阐释马克思主义,使其植根于中华优秀传统文化的土壤之中,具有中国特色、中国风格、中国气派。
\end{enumerate}

\section{马克思主义中国化时代化理论成果及其关系(重要)}

\begin{enumerate}[label=(\arabic*)]
\item 理论成果:在马克思主义中国化时代化的历史进程中,产生了毛泽东思想、邓小平理论、"三个代表"重要思想、科学发展观、习近平新时代中国特色社会主义思想。

\item 关系:马克思主义中国化时代化的理论成果是一脉相承又与时俱进的关系。

\begin{enumerate}[label=(\roman*)]
\item 一方面,毛泽东思想所蕴含的马克思主义的立场、观点和方法,为中国特色社会主义理论体系提供了基本遵循。

\item 另一方面,中国特色社会主义理论体系在新的历史条件下进一步丰富和发展了毛泽东思想。
\end{enumerate}
\end{enumerate}

\section{马克思主义中国化时代化的三次飞跃}

毛泽东思想、中国特色社会主义理论体系、习近平新时代中国特色社会主义思想。

\section{毛泽东思想形成发展的历史条件}

\begin{enumerate}[label=(\arabic*)]
\item 社会历史条件(国内背景):近代中国社会的特殊国情和中国革命的特殊性。

\item 时代背景(国际背景):19世纪末20世纪初,世界进入帝国主义和无产阶级革命时代,战争与革命成为时代主题。

\item 实践基础:中国共产党领导人民进行革命和建设的成功实践是毛泽东思想形成和发展的实践基础。
\end{enumerate}

\section{毛泽东思想活的灵魂(重要)}
它们有三个基本方面,即实事求是、群众路线、独立自主。

\section{实事求是}

\begin{enumerate}[label=(\arabic*)]
\item 含义:
\begin{enumerate}[label=(\roman*)]
\item 实事求是,就是一切从实际出发,理论联系实际,坚持在实践中检验真理和发展真理。

\item "实事"就是客观存在着的一切事物,"是"就是客观事物的内部联系,即规律性,"求"就是我们去研究。
\end{enumerate}

\item 重要性:实事求是,是马克思主义的根本观点,是中国共产人认识世界、改造世界的根本要求,是我们党的基本思想方法、工作方法、领导方法。
\end{enumerate}

\section{群众路线}
答:
\begin{enumerate}[label=(\arabic*)]
\item 含义:
\begin{enumerate}[label=(\roman*)]
\item 群众路线,就是一切为了群众,一切依靠群众,从群众中来,到群众中去,把党的正确主张变为群众的自觉行动。

\item 坚持群众路线,就是坚持全心全意为人民服务的根本宗旨。全心全意为人民服务,是我们党一切行动的根本出发点和落脚点,是我们党区别于其他一切政党的根本标志。
\end{enumerate}

\item 重要性:
\begin{enumerate}[label=(\roman*)]
\item 群众路线都是我们党的生命线和根本工作路线,是我们党永葆青春活力和战斗力的重要传家宝。

\item 群众路线本质上体现的是马克思主义关于人民群众是历史创造者这一基本原理。
\end{enumerate}
\end{enumerate}

\section{独立自主}
答:
\begin{enumerate}[label=(\arabic*)]
\item 含义:独立自主,就是坚持独立思考,走自己的路,就是坚定不移地维护民族独立、捍卫国家主权,把立足点放在依靠自己力量的基础上,同时积极争取外援,开展国际经济文化交流,学习外国一切对我们有益的先进事物。

\item 重要性:独立自主是中华民族的优良传统,是中国共产党、中华人民共和国立党立国的重要原则,是我们党从中国实际出发、依靠党和人民力量进行革命、建设、改革的必然结论、不论过去、现在和将来,我们都要把国家和民族的发展放在自己力量的基本点上,增强民族自尊心和自信心,坚定不移走自己的路。
\end{enumerate}

\section{毛泽东思想的历史地位}
答:
\begin{enumerate}[label=(\arabic*)]
\item 马克思主义中国化时代化的第一个重大理论成果;

\item 中国革命和建设的科学指南;

\item 中国共产党和中国人民宝贵的精神财富。
\end{enumerate}

\section{新民主主义革命总路线的内容}
答:
无产阶级领导的,人民大众的,反对帝国主义、封建主义和官僚资本主义的革命。

\section{新民主主义革命的对象}
答:
\begin{enumerate}[label=(\arabic*)]
\item 分清敌友,这是革命的首要问题。近代中国社会的性质和主要矛盾,决定了中国革命的主要敌人就是帝国主义、封建主义和官僚资本主义。

\item 帝国主义是中国革命的首要对象。

\item 封建地主阶级是帝国主义统治中国和封建军阀实行专制统治的社会基础。是中国经济现代化和政治民主化的主要的、直接的障碍。

\item 官僚资本主义是依靠帝国主义、勾结封建势力、利用国家政权力量而发展起来的买办的、封建的国家垄断资本主义。
\end{enumerate}

\section{新民主主义革命的动力}
答:
\begin{enumerate}[label=(\arabic*)]
\item 新民主主义革命的动力包括无产阶级、农民阶级、城市小资产阶级和民族资产阶级。

\item 无产阶级是中国革命最基本的动力。无产阶级是新的社会生产力的代表,是近代中国最进步的阶级,是中国革命的领导力量。

\item 农民是中国革命的主力军。他们深受帝国主义、封建主义和官僚资本主义的压迫和剥削,具有强烈的反帝反封建的革命要求。

\item 城市小资产阶级是无产阶级的可靠同盟者。城市小资产阶级,包括广大的知识分子、小商人、手工业者和自由职业者,同样受帝国主义、封建主义和官僚资本主义的压迫和掠夺。

\item 民族资产阶级也是中国革命的动力之一。半殖民半封建社会的民族资产阶级是一个带有两面性的阶级。它既不可能充当革命的主要力量,更不可能是革命的领导力量。
\end{enumerate}

\section{新民主主义的政治纲领}
答:
推翻帝国主义和封建主义的统治,建立一个无产阶级领导的、以工农联盟为基础的、各革命阶级联合专政的新民主主义的共和国。

\section{新民主主义的经济纲领}
答:
没收封建地主阶级的土地归农民所有,没收官僚资产阶级的垄断资本归新民主主义的国家所有,保护民族工商业。

\section{新民主主义的文化纲领}
答:
新民主主义的文化,就是无产阶级领导的人民大众的反帝反封建的文化,即民族的科学的大众的文化。

\section{新民主主义革命道路的内容}
答:
\begin{enumerate}[label=(\arabic*)]
\item 中国革命走农村包围城市、武装夺取政权的道路,根本在于处理好土地革命、武装斗争、农村革命根据地建设三者之间的关系。

\item 土地革命是中国革命的基本内容:武装斗争是中国革命的主要形式,是农村革命根据地建设和土地革命的强有力保证;农村革命根据地是中国革命的战略阵地,是进行武装斗争和开展土地革命的依托。
\end{enumerate}

\section{新民主主义革命的三大法宝}
答:
\begin{enumerate}[label=(\arabic*)]
\item 统一战线、武装斗争、党的建设是党在中国革命中战胜敌人的三个主要法宝。

\item 统一战线是无产阶级政党策略思想的重要内容。

\item 武装斗争是中国革命的特点和优点之一。

\item 中国共产党要领导革命取得胜利,必须不断加强党的思想建设、组织建设和作风建设。

\item 统一战线和武装斗争是中国革命的两个基本特点,是战胜敌人的两个基本武器。统一战线是实行武装斗争的统一战线,武装斗争是统一战线的中心支柱,党的组织则是掌握统一战线和武装斗争这两个武器以实行对敌冲锋陷阵的英勇战士。
\end{enumerate}

\section{新民主主义社会中国的五种经济成分}
答:
新民主主义社会不是一个独立的社会形态,而是由新民主主义向社会主义转变的过渡性社会形态。在新民主主义社会中,存在着五种经济成分,即社会主义性质的国营经济、半社会主义性质的合作社经济、农民和手工业者的个体经济、私人资本主义经济和国家资本主义经济。

\section{过渡时期的总路线}
答:
从中华人民共和国成立,到社会主义改造基本完成,这是一个过渡时期。党在这个过渡时期的总路线和总任务,是要在一个相当长的时期内,逐步实现国家的社会主义工业化,并逐步实现国家对农业、对手工业和对资本主义工商业的社会主义改造。

\section{党在过渡时期的主要内容}
答:
"一化三改"。"一化"即社会主义工业化,"三改"即对个体农业、手工业和资本主义工商业的社会主义改造。他么之间相互联系,不可分离,可以比喻为鸟的"主体"和"两翼"。其中,"一化"是"主体","三改"是"两翼"。

\section{社会主义改造的历史经验}
答:
\begin{enumerate}[label=(\arabic*)]
\item 坚持社会主义工业化建设与社会主义改造同时并举。

\item 采取积极引导、逐步过渡的方式。

\item 用和平方式进行改造。
\end{enumerate}

\section{《论十大关系》的基本方针}
答:
努力把党内外、国内外的一切积极的因素,直接的、间接的积极因素,全部调动起来。

\section{社会主义社会的矛盾}
答:
\begin{enumerate}[label=(\arabic*)]
\item 关于社会主义社会的基本矛盾。

\item 关于我国社会的主要矛盾和根本任务。

\item 关于社会主义社会两类不同性质的矛盾。

\item 关于正确处理两类不同性子社会矛盾的基本方法。

\item 关于正确处理人民内部矛盾的方针。

\item 关于严格区分两类不同性质矛盾和正确处理人民内部矛盾的目的和意义。
\end{enumerate}

\section{中国工业化道路的总方针}
答:
毛泽东提出了以农业为基础,以工业为主导,以弄轻重为序发展国民经济的总方针,以及以整套"两条腿走路"的工业化发展思路,即重工业和轻工业同时并举,中央工业和地方工业同时并举,沿海工业和内地工业同时并举,大型企业和中小型企业同时并举,等等。

\section{初步探索的意义}
答:
\begin{enumerate}[label=(\arabic*)]
\item 巩固和发展了我国的社会主义制度。

\item 为开创中国特色社会主义提供了宝贵经验、理论准备、物质基础。

\item 丰富了科学社会主义的理论和实践。
\end{enumerate}

\section{初步探索的经验与教训}
答:
\begin{enumerate}[label=(\arabic*)]
    \item 必须把马克思主义与中国实际相结合,探索符合中国特点的社会主义建设道路。
    \item 必须正确认识社会主义社会的主要矛盾和根本任务,集中力量发展生产力。
    \item 必须从实际出发进行社会主义建设,建设规模和速度要与国力相适应,不能急于求成。
    \item 必须发展社会主义民主,健全社会主义法制。
    \item 必须坚持党的民主集中制和集体领导制度,加强执政党建设。
    \item 必须坚持对外开放,借鉴和吸收人类文明成果建设社会主义,不能关起门来搞建设。
\end{enumerate}

\section{中国特色社会主义理论体系形成发展过程}
答:
\begin{enumerate}[label=(\arabic*)]
\item 邓小平理论标志着中国特色社会主义制度理论体系的形成。

\item "三个代表重要思想"是中国特色社会主义理论体系的跨世纪发展。

\item 科学发展观是中国特色社会主义理论体系在新世纪新阶段的新发展。

\item 习近平新时代中国特色社会主义是中国特色社会主义理论特细在新时代的新篇章。
\end{enumerate}

\section{社会主义的本质}
答:
\begin{enumerate}[label=(\arabic*)]
\item 社会主义的本质,是解放生产力,消灭剥削,消除两极分化,最终达到共同富裕。

\item 邓小平对社会主义本质的概括,既包括了社会主义社会的生产力问题,又包括了社会主义社会的生产关系问题,是一个有机的整体。

\begin{enumerate}[label=(\roman*)]
\item 首先,它突出强调解放和发展生产力在社会主义社会发展中的重要地位,纠正了过去关于发展生产力的一些错误观念,反映了中国社会主义整个历史阶段尤其是初级阶段特别需要注重生产力发展的迫切要求,并明确了社会主义基本制度建立后还要通过改革进一步解放生产力。

\item 其次,它突出地强调"消灭剥削,消除两极分化,最终达到共同富裕",从生产关系和发展目标角度认识和把握社会主义本质。
\end{enumerate}
\end{enumerate}

\section{邓小平理论的精髓}
答:
\begin{enumerate}[label=(\arabic*)]
\item 解放思想、实事求是是邓小平理论的精髓。

\item 解放思想、实事求是贯穿邓小平理论形成发展的全过程。邓小平深刻阐明了解放思想和实事求是的辩证统一关系,只有解放思想才能达到实事求是,只有实事求是才是真正的解放思想。
\end{enumerate}

\section{社会主义初级阶段}
答:
\begin{enumerate}[label=(\arabic*)]
\item 党的十三大召开前夕,邓小平指出:"我们党的十三大要阐述中国社会主义是处在一个什么阶段,就是处在初级阶段,是初级阶段的社会主义。社会主义本身是共产主义的初级阶段,而我们中国又处在社会主义的初级阶段,就是不发达的阶段。一切都要从这个实际出发,根据这个实际来制订规划。"

\item 党的十三大系统地阐述了社会主义初级阶段的科学内涵。首先,阐明了社会主义初级阶段这个论断包括两层含义:

\begin{enumerate}[label=(\roman*)]
\item 第一,我国社会已经是社会主义社会。我们必须坚持而不能离开社会主义。

\item 第二,我国的社会主义社会还处在初级阶段。我们必须从这个实际出发,而不能超越这个阶段。
\end{enumerate}

\item 其次,强调了社会主义初级阶段的长期性。

\item 最后,阐述了社会主义初级阶段的基本特征。
\end{enumerate}

\section{社会主义改革开放理论(为什么改革是中国第二次革命?)}
答:
\begin{enumerate}[label=(\arabic*)]
\item 性质:
\begin{enumerate}[label=(\roman*)]
\item 改革是一场深刻的社会变革,是中国的第二次革命,是实现中国现代化的必由之路。

\item 改革不是原有经济体制的细枝末节的修补,它的实质和目标是要从根本上改变束缚我国生产力发展的经济体制,建立充满生机和活力的社会主义新经济体制,同时相应地改革政治体制和其他方面的体制,以实现中国的社会主义现代化。

\item 改革不是一个阶级推翻另一个阶级那种原来意义上的革命,不是也不允许否定和抛弃我们建立起来的社会主义基本制度,而是社会主义制度的自我完善和发展。
\end{enumerate}

\item 改革是社会主义社会发展的直接动力。改革是从根本上改变束缚生产力发展的经济体制,促进生产力的发展,解决社会主义社会发展的动力问题。

\item 改革是一项崭新的事业,是一个大试验。不能因循守旧,四平八稳,不能不顾条件,急于求成。判断改革和各方面是非得失,归根到底,主要看是否符合"三个有利于"标准。
\end{enumerate}

\section{"三个代表"重要思想的核心观点}
答:
\begin{enumerate}[label=(\arabic*)]
\item 始终代表中国先进生产力的发展要求,就是党的理论、路线、纲领、方阵、政策和各项工作,体现不断推动社会生产力的解放和发展的要求,通过发展生产力不断提高人民群众的生活水平。

\item 始终代表中国先进文化的前进方向,就是党的理论、路线、纲领、方阵、政策和各项工作,必须努力体现发展面向现代化、面向世界、面向未来的,民族的科学的大众的社会主义文化的要求。为我国经济发展和社会进步提供精神动力和智力支持。

\item 始终代表中国最广大人民的根本利益,就是党的理论、路线、纲领、方阵、政策和各项工作,必须坚持把人民的根本利益作为出发点和归宿,充分发挥人民群众的积极性主动性创造性,使人民群众不断获得切实的经济、政治、文化利益。
\end{enumerate}


\section{和平共处五项原则}
\begin{enumerate}[label=(\arabic*)]
    
\item 互相尊重主权和领土完整、互不侵犯、互不干涉内政、平等互利。

\end{enumerate}







\chapter{习概期末大题}



\end{document}