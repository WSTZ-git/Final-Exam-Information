\documentclass[12pt, a4paper, oneside]{ctexbook}
% 全角符号转换为半角符号
\usepackage{newunicodechar}
\newunicodechar{。}{.}
\newunicodechar{,}{,}
\newunicodechar{:}{:}
\newunicodechar{;}{;}
\newunicodechar{!}{!}
\newunicodechar{(}{(}
\newunicodechar{)}{)}
\usepackage{amsmath, amsthm, amssymb, bm, graphicx, enumitem, hyperref, mathrsfs}

\title{{\Huge{\textbf{思政期末大题汇总}}}}
\author{}
\date{\today}
\linespread{1.5}
\newtheorem{theorem}{定理}[section]
\newtheorem{definition}[theorem]{定义}
\newtheorem{lemma}[theorem]{引理}
\newtheorem{corollary}[theorem]{推论}
\newtheorem{example}[theorem]{例}
\newtheorem{proposition}[theorem]{命题}

\begin{document}

\maketitle

\pagenumbering{roman}
\setcounter{page}{1}

% \begin{center}
%     \Huge\textbf{前言}
% \end{center}~\


% ~\\
% \begin{flushright}
%     \begin{tabular}{c}
%         邹文杰\\
%         \today
%     \end{tabular}
% \end{flushright}

% \newpage
\pagenumbering{Roman}
\setcounter{page}{1}
\tableofcontents
\newpage
\setcounter{page}{1}
\pagenumbering{arabic}

\chapter{毛概期末大题}

\section{马克思主义中国化时代化的必要性(重点)}

马克思主义中国化时代化,就是立足中国国情和时代特点,坚持把马克思主义基本原理同中国具体实际相结合、同中华优秀传统文化相结合。

\begin{enumerate}
\item 第一个原因:推进马克思主义中国化时代化,是马克思主义理论本身发展的内在要求。作为洞察时代、引领时代的科学理论,马克思主义只有正确运用于实践并在实践中不断发展才能体现其科学性,彰显其强大力量。马克思主义只有实现中国化时代化,才能不断发展自身,始终保持蓬勃生机和旺盛活力。
\item 第二个原因:推进马克思主义中国化时代化,是解决中国实际问题的客观需要。马克思主义要在中国发挥指导作用,就必须中国化时代化。只有与中国国情相结合,与时代发展同进步,马克思主义才能真正解决中国的实际问题。
\end{enumerate}

\section{马克思主义中国化时代化的科学内涵}

\begin{enumerate}
\item 运用马克思主义的立场、观点和方法,观察时代、把握时代、引领时代,解决中国革命、建设、改革中的实际问题。

\item 总结和提炼中国革命、建设、改革的实践经验并将其上升为理论,不断丰富和发展马克思主义的理论宝库,赋予马克思主义以新的时代内涵。

\item 运用中国人民喜闻乐见的民族语言来阐释马克思主义,使其植根于中华优秀传统文化的土壤之中,具有中国特色、中国风格、中国气派。
\end{enumerate}

\section{马克思主义中国化时代化理论成果及其关系(重点)}

\begin{enumerate}
\item 理论成果:在马克思主义中国化时代化的历史进程中,产生了毛泽东思想、邓小平理论、"三个代表"重要思想、科学发展观、习近平新时代中国特色社会主义思想。

\item 关系:马克思主义中国化时代化的理论成果是一脉相承又与时俱进的关系。

\begin{enumerate}[label=(\arabic*)]
\item 一方面,毛泽东思想所蕴含的马克思主义的立场、观点和方法,为中国特色社会主义理论体系提供了基本遵循。

\item 另一方面,中国特色社会主义理论体系在新的历史条件下进一步丰富和发展了毛泽东思想。
\end{enumerate}
\end{enumerate}

\section{毛泽东思想活的灵魂(重点)}
它们有三个基本方面,即实事求是、群众路线、独立自主。实事求是是毛泽东思想的精髓。

\section{实事求是(重点)}

\begin{enumerate}
\item 含义:
\begin{enumerate}[label=(\arabic*)]
\item 实事求是,就是一切从实际出发,理论联系实际,坚持在实践中检验真理和发展真理。

\item “实事”就是客观存在着的一切事物,“是”就是客观事物的内部联系,即规律性,“求”就是我们去研究。
\end{enumerate}

\item 重要性:实事求是,是马克思主义的根本观点,是中国共产党人认识世界、改造世界的根本要求,是我们党的基本思想方法、工作方法、领导方法。
\end{enumerate}

\section{群众路线}

\begin{enumerate}
\item 含义:
\begin{enumerate}[label=(\arabic*)]
\item 群众路线,就是一切为了群众,一切依靠群众,从群众中来,到群众中去,把党的正确主张变为群众的自觉行动。

\item 坚持群众路线,就是坚持全心全意为人民服务的根本宗旨。全心全意为人民服务,是我们党一切行动的根本出发点和落脚点,是我们党区别于其他一切政党的根本标志。
\end{enumerate}

\item 重要性:
\begin{enumerate}[label=(\arabic*)]
\item 群众路线是我们党的生命线和根本工作路线,是我们党永葆青春活力和战斗力的重要传家宝。

\item 群众路线本质上体现的是马克思主义关于人民群众是历史创造者这一基本原理。
\end{enumerate}
\end{enumerate}

\section{独立自主}

\begin{enumerate}
\item 含义:独立自主,就是坚持独立思考,走自己的路,就是坚定不移地维护民族独立、捍卫国家主权,把立足点放在依靠自己力量的基础上,同时积极争取外援,开展国际经济文化交流,学习外国一切对我们有益的先进事物。

\item 重要性:独立自主是中华民族的优良传统,是中国共产党、中华人民共和国立党立国的重要原则,是我们党从中国实际出发、依靠党和人民力量进行革命、建设、改革的必然结论、不论过去、现在和将来,我们都要把国家和民族的发展放在自己力量的基本点上,增强民族自尊心和自信心,坚定不移走自己的路。
\end{enumerate}

\section{毛泽东思想的历史地位(重点)}

\begin{enumerate}
\item 马克思主义中国化时代化的第一个重大理论成果;

\item 中国革命和建设的科学指南;

\item 中国共产党和中国人民宝贵的精神财富。
\end{enumerate}

\section{新民主主义革命总路线的内容(重点)}

无产阶级领导的,人民大众的,反对帝国主义、封建主义和官僚资本主义的革命。

\section{新民主主义革命的对象(重点)}

\begin{enumerate}
\item 分清敌友,这是革命的首要问题。近代中国社会的性质和主要矛盾,决定了中国革命的主要敌人就是帝国主义、封建主义和官僚资本主义。

\item 帝国主义是中国革命的首要对象。

\item 封建地主阶级是帝国主义统治中国和封建军阀实行专制统治的社会基础。是中国经济现代化和政治民主化的主要的、直接的障碍。

\item 官僚资本主义是依靠帝国主义、勾结封建势力、利用国家政权力量而发展起来的买办的、封建的国家垄断资本主义。
\end{enumerate}

\section{新民主主义革命的动力}

\begin{enumerate}
\item 新民主主义革命的动力包括无产阶级、农民阶级、城市小资产阶级和民族资产阶级。

\item 无产阶级是中国革命最基本的动力。无产阶级是新的社会生产力的代表,是近代中国最进步的阶级,是中国革命的领导力量。

\item 农民是中国革命的主力军。他们深受帝国主义、封建主义和官僚资本主义的压迫和剥削,具有强烈的反帝反封建的革命要求。

\item 城市小资产阶级是无产阶级的可靠同盟者。城市小资产阶级,包括广大的知识分子、小商人、手工业者和自由职业者,同样受帝国主义、封建主义和官僚资本主义的压迫和掠夺。

\item 民族资产阶级也是中国革命的动力之一。半殖民半封建社会的民族资产阶级是一个带有两面性的阶级。它既不可能充当革命的主要力量,更不可能是革命的领导力量。
\end{enumerate}

\section{新民主主义的政治纲领(重点)}

推翻帝国主义和封建主义的统治,建立一个无产阶级领导的、以工农联盟为基础的、各革命阶级联合专政的新民主主义的共和国。

\section{新民主主义的经济纲领(重点)}

没收封建地主阶级的土地归农民所有,没收官僚资产阶级的垄断资本归新民主主义的国家所有,保护民族工商业。

\section{新民主主义的文化纲领}

新民主主义的文化,就是无产阶级领导的人民大众的反帝反封建的文化,即民族的科学的大众的文化。

\section{新民主主义革命道路的内容(重点)}

\begin{enumerate}
\item 中国革命走农村包围城市、武装夺取政权的道路,根本在于处理好土地革命、武装斗争、农村革命根据地建设三者之间的关系。

\item 土地革命是中国革命的基本内容;

\item 武装斗争是中国革命的主要形式,是农村革命根据地建设和土地革命的强有力保证;

\item 农村革命根据地是中国革命的战略阵地,是进行武装斗争和开展土地革命的依托。
\end{enumerate}

\section{新民主主义革命的三大法宝(重点)}

\begin{enumerate}
\item 统一战线、武装斗争、党的建设是党在中国革命中战胜敌人的三个主要法宝。

\item 统一战线是无产阶级政党策略思想的重要内容。

\item 武装斗争是中国革命的特点和优点之一。

\item 中国共产党要领导革命取得胜利,必须不断加强党的思想建设、组织建设和作风建设。

\item 统一战线和武装斗争是中国革命的两个基本特点,是战胜敌人的两个基本武器。统一战线是实行武装斗争的统一战线,武装斗争是统一战线的中心支柱,党的组织则是掌握统一战线和武装斗争这两个武器以实行对敌冲锋陷阵的英勇战士。
\end{enumerate}

\section{新民主主义社会中国的五种经济成分}

新民主主义社会不是一个独立的社会形态,而是由新民主主义向社会主义转变的过渡性社会形态。在新民主主义社会中,存在着五种经济成分,即社会主义性质的国营经济、半社会主义性质的合作社经济、农民和手工业者的个体经济、私人资本主义经济和国家资本主义经济。

\section{过渡时期的总路线(重点)}

从中华人民共和国成立,到社会主义改造基本完成,这是一个过渡时期。党在这个过渡时期的总路线和总任务,是要在一个相当长的时期内,逐步实现国家的社会主义工业化,并逐步实现国家对农业、对手工业和对资本主义工商业的社会主义改造。

\section{社会主义改造的历史经验}

\begin{enumerate}
\item 坚持社会主义工业化建设与社会主义改造同时并举。

\item 采取积极引导、逐步过渡的方式。

\item 用和平方法进行改造。
\end{enumerate}

\section{《论十大关系》的基本方针(重点)}

努力把党内外、国内外的一切积极的因素,直接的、间接的积极因素,全部调动起来。

\section{中国工业化道路的总方针(重点)}

以农业为基础,以工业为主导,以农轻重为序发展国民经济的总方针,以及一整套“两条腿走路”的工业化发展思路,即重工业和轻工业同时并举,中央工业和地方工业同时并举,沿海工业和内地工业同时并举,大型企业和中小型企业同时并举,等等。

\section{中国特色社会主义理论体系形成发展过程(重点)}

\begin{enumerate}
\item 邓小平理论是中国特色社会主义理论体系的形成。

\item “三个代表重要思想”是中国特色社会主义理论体系的跨世纪发展。

\item 科学发展观是中国特色社会主义理论体系在新世纪新阶段的新发展。

\item 习近平新时代中国特色社会主义思想是中国特色社会主义理论体系在新时代的新篇章。
\end{enumerate}

\section{邓小平理论首要的基本的理论问题}

在中国这样一个经济文化比较落后的国家建设什么样的社会主义、怎样建设社会主义是一个首要的基本的理论问题.

\section{社会主义的本质(重点)}

\begin{enumerate}
\item 社会主义的本质,是解放生产力,消灭剥削,消除两极分化,最终达到共同富裕。

\item 邓小平对社会主义本质的概括,既包括了社会主义社会的生产力问题,又包括了社会主义社会的生产关系问题,是一个有机的整体。

\begin{enumerate}[label=(\arabic*)]
\item 首先,它突出强调解放和发展生产力在社会主义社会发展中的重要地位,纠正了过去关于发展生产力的一些错误观念,反映了中国社会主义整个历史阶段尤其是初级阶段特别需要注重生产力发展的迫切要求,并明确了社会主义基本制度建立后还要通过改革进一步解放生产力。

\item 其次,它突出地强调“消灭剥削,消除两极分化,最终达到共同富裕”,从生产关系和发展目标角度认识和把握社会主义本质。
\end{enumerate}
\end{enumerate}

\section{邓小平理论的精髓}

\begin{enumerate}
\item 解放思想、实事求是是邓小平理论的精髓。

\item 解放思想、实事求是贯穿邓小平理论形成发展的全过程。邓小平深刻阐明了解放思想和实事求是的辩证统一关系,只有解放思想才能达到实事求是,只有实事求是才是真正的解放思想。
\end{enumerate}

\section{社会主义初级阶段}

\begin{enumerate}
\item 我们党的十三大要阐述中国社会主义是处在一个什么阶段,就是处在初级阶段,是初级阶段的社会主义。社会主义本身是共产主义的初级阶段,而我们中国又处在社会主义的初级阶段,就是不发达的阶段。一切都要从这个实际出发,根据这个实际来制订规划。

\item 党的十三大系统地阐述了社会主义初级阶段的科学内涵。首先,阐明了社会主义初级阶段这个论断包括两层含义:

\begin{enumerate}[label=(\arabic*)]
\item 第一,我国社会已经是社会主义社会。我们必须坚持而不能离开社会主义。

\item 第二,我国的社会主义社会还处在初级阶段。我们必须从这个实际出发,而不能超越这个阶段。
\end{enumerate}

\item 其次,强调了社会主义初级阶段的长期性。

\item 最后,阐述了社会主义初级阶段的基本特征。
\end{enumerate}

\section{社会主义改革开放理论(为什么改革是中国第二次革命?)(重点)}

\begin{enumerate}
\item 性质:
\begin{enumerate}[label=(\arabic*)]
\item 改革是一场深刻的社会变革,是中国的第二次革命,是实现中国现代化的必由之路。

\item 改革不是原有经济体制的细枝末节的修补,它的实质和目标是要从根本上改变束缚我国生产力发展的经济体制,建立充满生机和活力的社会主义新经济体制,同时相应地改革政治体制和其他方面的体制,以实现中国的社会主义现代化。

\item 改革不是一个阶级推翻另一个阶级那种原来意义上的革命,不是也不允许否定和抛弃我们建立起来的社会主义基本制度,而是社会主义制度的自我完善和发展。
\end{enumerate}

\item 作用:改革是社会主义社会发展的直接动力。改革是从根本上改变束缚生产力发展的经济体制,促进生产力的发展,解决社会主义社会发展的动力问题。

\item 判断标准:改革是一项崭新的事业,是一个大试验。不能因循守旧,四平八稳,不能不顾条件,急于求成。判断改革和各方面是非得失,归根到底,主要看是否符合“三个有利于”标准。
\end{enumerate}

\section{“三个代表”重要思想的核心观点(重点)}

\begin{enumerate}
\item 始终代表中国先进生产力的发展要求,就是党的理论、路线、纲领、方针、政策和各项工作,体现不断推动社会生产力的解放和发展的要求,通过发展生产力不断提高人民群众的生活水平。

\item 始终代表中国先进文化的前进方向,就是党的理论、路线、纲领、方针、政策和各项工作,必须努力体现发展面向现代化、面向世界、面向未来的,民族的科学的大众的社会主义文化的要求。为我国经济发展和社会进步提供精神动力和智力支持。

\item 始终代表中国最广大人民的根本利益,就是党的理论、路线、纲领、方针、政策和各项工作,必须坚持把人民的根本利益作为出发点和归宿,充分发挥人民群众的积极性主动性创造性,使人民群众不断获得切实的经济、政治、文化利益。
\end{enumerate}

\section{和平共处五项原则(重点)}

互相尊重主权和领土完整、互不侵犯、互不干涉内政、平等互利。

\section{如何正确认识和处理改革、发展、稳定的关系}

\begin{enumerate}
\item 改革是动力.没有改革,就不可能走出一条建设中国特色社会主义的正确道路,我们的事业就不可能顺利前进。

\item 发展是目的。没有发展,就不可能实现现代化,也就不可能保持党和国家的长治久安。

\item 稳定是前提。没有稳定,改革和发展都无法进行。

\item 要把改革的力度、发展的速度和社会可承受的程度统一起来,把不断改善人民生活作为处理改革发展稳定关系的重要结合点。在社会稳定中推进改革发展,通过改革发展促进社会稳定。
\end{enumerate}

\section{科学发展观的科学内涵是什么?}
\begin{enumerate}
\item 推动经济社会发展是科学发展观的第一要义;
\item 以人为本是科学发展观的核心立场;
\item 全面协调可持续是科学发展观的基本要求;
\item 统筹兼顾是科学发展观的根本方法.
\end{enumerate}

\section{科学发展观的基本要求}

全面协调可持续。
\begin{enumerate}
\item “全面”是指发展要有全面性、整体性,不仅经济要发展,而且各个方面都要发展。

\item “协调”是指发展要有协调性、均衡性,各个方面、各个环节的发展要相互适应、相互促进。

\item “可持续”是指发展要有持久性、连续性,不仅当前要发展,而且要保证长远发展。
\end{enumerate}

\section{构建社会主义和谐社会的总要求?(重点)}

民主法治,公平正义,诚信友爱,充满活力,安定有序,人与自然和谐相处,是构建社会主义和谐社会的总要求。


\chapter{习概期末大题}

\section{“两个结合”}

\begin{enumerate}
\item 习近平新时代中国特色社会主义思想是马克思主义基本原理同中国具体实际相结合、同中华优秀传统文化相结合的重大成果.

\item “两个结合”是对坚持和发展马克思主义作出的重大理论贡献.

\item 习近平新时代中国特色社会主义思想坚持把马克思主义基本原理同中国具体实际相结合,用马克思主义之“矢”去射新时代中国之“的”。

\item 习近平新时代中国特色社会主义思想坚持把马克思主义基本原理同中华优秀传统文化相结合,不断夯实马克思主义中国化时代化的历史基础和群众基础.
\end{enumerate}

\section{习近平新时代中国特色社会主义思想的主要内容}

习近平新时代中国特色社会主义思想内涵十分丰富,党的十九大、十九届六中全会提出的“十个明确”“十四个坚持”“十三个方面成就”概括了习近平新时代中国特色社会主义思想的主要内容。

\section{习近平新时代中国特色社会主义思想的世界观、方法论}

\begin{enumerate}
\item 党的二十大提出的“六个必须坚持”,是习近平新时代中国特色社会主义思想的世界观、方法论和贯穿其中的立场观点方法的重要体现。

\item “六个必须坚持”,就是必须坚持人民至上、必须坚持自信自立、必须坚持守正创新、必须坚持问题导向、必须坚持系统观念、必须坚持胸怀天下。
\end{enumerate}

\section{“五位一体”}

中国特色社会主义事业总体布局包括经济建设、政治建设、文化建设、社会建设、生态文明建设,“五位一体”.

\section{“四个全面”}

“四个全面”,就是全面深化改革、全面依法治国、全面从严治党、全面建设社会主义现代化国家.

\section{“中国梦的本质”}

国家富强、民族复兴、人民幸福。

\section{中国式现代化的中国特色}

\begin{enumerate}
\item 中国式现代化是人口规模巨大的现代化。

\item 中国式现代化是全体人民共同富裕的现代化。

\item 中国式现代化是物质文明和精神文明相协调的现代化。

\item 中国式现代化是人与自然和谐共生的现代化。

\item 中国式现代化是走和平发展道路的现代化。
\end{enumerate}

\section{中国式现代化的本质要求}

中国式现代化的本质要求是:坚持中国共产党领导,坚持中国特色社会主义,实现高质量发展,发展全过程人民民主,丰富人民精神世界,实现全体人民共同富裕,促进人与自然和谐共生,推动构建人类命运共同体,创造人类文明新形态。

\section{推进中国式现代化需要牢牢把握的重大原则}

\begin{enumerate}
\item 坚持和加强党的全面领导。

\item 坚持中国特色社会主义道路。

\item 坚持以人民为中心的发展思想。

\item 坚持深化改革开放。

\item 坚持发扬斗争精神。
\end{enumerate}

\section{扎实推进共同富裕,必须坚持什么原则和思路?}

\begin{enumerate}
\item 原则:要把握好鼓励勤劳创新致富、坚持基本经济制度、尽力而为量力而行、坚持循序渐进的原则。

\item 思路:坚持以人民为中心的发展思想,在高质量发展中促进共同富裕,正确处理效率和公平的关系,构建初次分配、再分配、三次分配协调配套的基础性制度安排,加大税收、社保、转移支付等调节力度并提高精准性,扩大中等收入群体比重,增加低收入群体收入,合理调节高收入,取缔非法收入,形成中间大、两头小的橄榄型分配结构,促进社会公平正义,促进人的全面发展,使全体人民朝着共同富裕目标扎实迈进。
\end{enumerate}

\section{全面深化改革开放的正确方向}

\begin{enumerate}
\item 必须坚持和改善党的全面领导、坚持和完善中国特色社会主义制度。

\item 必须坚持以人民为中心,促进社会公平正义、增进人民福祉。

\item 必须有利于进一步解放思想、进一步解放和发展社会生产力、进一步解放和增强社会活力。
\end{enumerate}

\section{坚持全面深化改革总目标}

\begin{enumerate}
\item 全面深化改革总目标是:完善和发展中国特色社会主义制度、推进国家治理体系和治理能力现代化。

\item “完善和发展中国特色社会主义制度”,规定了改革的根本方向,就是无论改什么、怎么改,都要坚持中国共产党领导、坚持中国特色社会主义,就是要通过改革推动中国特色社会主义制度更加成熟更加定型、更好发挥中国特色社会主义制度的优越性。

\item “推进国家治理体系和治理能力现代化”,明确了改革的鲜明指向和时代要求,就是要通过改革进一步增强我国制度活力,把制度优势转化为国家治理效能。
\end{enumerate}

\section{全面深化改革开放要坚持的正确方法论}
\begin{enumerate}
\item 增强全面深化改革的系统性、整体性、协同性。

\item 加强顶层设计和摸着石头过河相结合。

\item 统筹改革发展稳定。

\item 胆子要大,步子要稳。

\item 坚持重大改革于法有据。
\end{enumerate}

\section{新发展理念}

坚定不移贯彻创新、协调、绿色、开放、共享的新发展理念。

\section{如何把握新发展理念}

\begin{enumerate}
\item 要从根本宗旨上把握新发展理念。

\item 要从问题导向上把握新发展理念。

\item 要从忧患意识上把握新发展理念。
\end{enumerate}

\section{社会主义基本经济制度}

党中央对社会主义基本经济制度作出新概括,将公有制为主体、多种所有制经济共同发展,按劳分配为主体、多种分配方式并存,社会主义市场经济体制等共同作为社会主义基本经济制度。

\section{构建新发展格局}

构建新发展格局是以国内大循环为主体、国内国际双循环相互促进。

\section{如何推动构建新发展格局}

\begin{enumerate}
\item 着力推动实施扩大内需战略同深化供给侧结构性改革有机结合。

\item 着力发展实体经济。

\item 着力加快科技自立自强。

\item 着力推动产业链供应链优化升级。
\end{enumerate}

\section{深入实施科教兴国战略、人才强国战略、创新驱动发展战略}

\begin{enumerate}
\item 科教兴国战略,就是要全面落实科学技术是第一生产力的思想,坚持教育优先发展,把科技和教育作为经济社会发展的重中之重,促进教育同经济、科技的密切结合,把经济建设转到依靠科技进步和提高劳动者素质的轨道上来,为实现经济发展提供科技支撑和人才保障。科教兴国是我国的基本国策。实施科教兴国战略,要把教育摆在优先发展的战略地位,把科技创新作为提高社会生产力和综合国力的战略支撑,完善党对教育、科技工作的领导体制机制,不断优化我国科教事业发展总体布局,深化科教体制改革,促进科技和教育成果向现实生产力转化。

\item 人才强固战略,就是要牢固树立人才资源是第一资源的理念,把人才队伍建设提升到国家战略的高度,营造良好人才创新生态环境,充分发挥各类人才的积极性、主动性和创造性,开创人才辈出、人尽其才的新局面,把我国由人口大国转化为人才资源强国。人才强国战略是国家和民族长远发展大计。实施人才强国战略,要坚持党管人才原则,坚持尊重劳动、尊重知识、尊重人才、尊重创造,坚持各方面人才一起抓,实施更加积极、更加开放、更加有效的人才政策,完善人才战略布局,深化人才发展体制机制改革,加强国际人才交流,把各方面优秀人才集聚到党和人民事业中来。

\item 创新驱动发展战略,就是要坚持创新是第一动力,把创新驱动落实到现代化建设整个进程和各个方面,以创新推动经济转型发展,全面提升创新能力和效率,把创新发展主动权牢牢掌握在自己手中。创新在国家发展全局中居于核心位置。实施创新驱动发展战略,要坚持面向世界科技前沿、面向经济主战场、面向国家重大需求、面向人民生命健康,以国家战略需求为导向,坚持科技创新和体制机制创新双轮驱动,增强自主创新能力,集聚力量进行原创性引领性科技攻关,坚决打赢关键核心技术攻坚战,让创新成为全社会的共同行动。
\end{enumerate}

\section{中国特色社会主义政治制度}

\begin{enumerate}
\item 中国特色社会主义政治制度是由根本政治制度、基本政治制度、重要政治制度等组成的制度体系。

\item 人民代表大会制度是坚持党的领导、人民当家作主、依法治国有机统一的根本政治制度安排。

\item 中国共产党领导的多党合作和政治协商制度、民族区域自治制度以及基层群众自治制度构成我国的基本政治制度。

\item 在根本政治制度、基本政治制度基础之上,形成了一系列具有中国特色的重要政治制度。
\end{enumerate}

\section{坚持和拓展中国特色社会主义法治道路必须坚持什么原则?}

\begin{enumerate}
\item 坚持中国共产党领导。

\item 坚持以人民为中心。

\item 坚持法律面前人人平等。

\item 坚持依法治国和以德治国相结合。

\item 坚持从中国实际出发。
\end{enumerate}

\section{中国特色社会主义法制体系基本框架}
\begin{enumerate}
\item 完备的法律规范体系

\item 高效的法制实施体系。

\item 严密的法制监督体系。

\item 有力的法制保障体系。

\item 完备的党内法规体系。
\end{enumerate}

\section{坚定中国特色社会主义文化自信}

\begin{enumerate}
\item 文化自信使更基础、更广泛、更深厚的自信,是一个国家、一个民族发展中最基本、最深沉、最持久的力量。

\item 中华优秀传统文化是我们坚定文化自信的深厚基础。

\item 党带领人民在伟大斗争中孕育的革命文化和社会主义先进文化是我们坚定文化自信的坚强基石。

\item 中国特色社会主义伟大实践是我们坚定文化自信的现实基础。
\end{enumerate}

\section{生命共同体}

人的命脉在田,田的命脉在水,水的命脉在山,山的命脉在土,土的命脉在林和草,这个生命共同体是人类生存发展的物质基础。

\section{区分听党指挥、能打胜仗、作风优良}
\begin{enumerate}
\item 听党指挥是灵魂,决定军队建设的政治方向

\item 能打胜仗是核心,反映军队的根本职能和军队建设的根本指向。

\item 作风优良是保证,关系军队的性质、宗旨、本色。

\end{enumerate}

\section{构建人类命运共同体}

构建人类命运共同体,就是要携手世界各国人民共同建设持久和平、普遍安全、共同繁荣、开放包容、清洁美丽的世界。

\section{如何推动构建人类命运共同体?}

推动构建人类命运共同体,
\begin{enumerate}
\item 要弘扬全人类共同价值,

\item 落实全球发展倡议、全球安全倡议、全球文明倡议,

\item 广泛凝聚共识、汇聚力量,携手建设合作共赢的美好世界。
\end{enumerate}






























\end{document}
