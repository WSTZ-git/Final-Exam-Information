\documentclass[12pt, a4paper, oneside]{ctexbook}
% 全角符号转换为半角符号
\usepackage{newunicodechar}
\newunicodechar{。}{.}
\newunicodechar{,}{,}
\newunicodechar{:}{:}
\newunicodechar{;}{;}
\newunicodechar{!}{!}
\newunicodechar{(}{(}
\newunicodechar{)}{)}
\usepackage{amsmath, amsthm, amssymb, bm, graphicx, enumitem, hyperref, mathrsfs}

\title{{\Huge{\textbf{思政期末大题汇总}}}}
\author{邹文杰}
\date{\today}
\linespread{1.5}
\newtheorem{theorem}{定理}[section]
\newtheorem{definition}[theorem]{定义}
\newtheorem{lemma}[theorem]{引理}
\newtheorem{corollary}[theorem]{推论}
\newtheorem{example}[theorem]{例}
\newtheorem{proposition}[theorem]{命题}

\begin{document}

\maketitle

\pagenumbering{roman}
\setcounter{page}{1}

% \begin{center}
%     \Huge\textbf{前言}
% \end{center}~\


% ~\\
% \begin{flushright}
%     \begin{tabular}{c}
%         邹文杰\\
%         \today
%     \end{tabular}
% \end{flushright}

% \newpage
\pagenumbering{Roman}
\setcounter{page}{1}
\tableofcontents
\newpage
\setcounter{page}{1}
\pagenumbering{arabic}

\chapter{毛概期末大题}

\section{马克思主义中国化时代化的必要性(重要)}

马克思主义中国化时代化,就是立足中国国情和时代特点,坚持把马克思主义基本原理同中国具体实际相结合、同中华优秀传统文化系相结合。

\begin{enumerate}[label=(\arabic*)]
\item 第一个原因:推进马克思主义中国化时代化,是马克思主义理论本身发展的内在要求。作为洞察时代、引领时代的科学理论,马克思主义只有正确运用于实践并在实践中不断发展才能体现其科学性,彰显其强大力量。马克思主义只有实现中国化时代化,才能不断发展自身,始终保持蓬勃生机和旺盛活力。
\item 第二个原因:推进马克思主义中国化时代化,是解决中国实际问题的客观需要。马克思主义要在中国发挥指导作用,就必须中国化时代化。只有与中国国情相结合,与时代发展同进步,马克思主义才能真正解决中国的实际问题。
\end{enumerate}

\section{马克思主义中国化时代化的科学内涵}

\begin{enumerate}[label=(\arabic*)]
\item 运用马克思主义的立场、观点和方法,观察时代、把握时代、引领时代,解决中国革命、建设、改革中的实际问题。

\item 总结和提炼中国革命、建设、改革的实践经验并将其上升为理论,不断丰富和发展马克思主义的理论宝库,赋予马克思主义以新的时代内涵。

\item 运用中国人民喜闻乐见的民族语言来阐释马克思主义,使其植根于中华优秀传统文化的土壤之中,具有中国特色、中国风格、中国气派。
\end{enumerate}

\section{马克思主义中国化时代化理论成果及其关系(重要)}

\begin{enumerate}[label=(\arabic*)]
\item 理论成果:在马克思主义中国化时代化的历史进程中,产生了毛泽东思想、邓小平理论、"三个代表"重要思想、科学发展观、习近平新时代中国特色社会主义思想。

\item 关系:马克思主义中国化时代化的理论成果是一脉相承又与时俱进的关系。

\begin{enumerate}[label=(\roman*)]
\item 一方面,毛泽东思想所蕴含的马克思主义的立场、观点和方法,为中国特色社会主义理论体系提供了基本遵循。

\item 另一方面,中国特色社会主义理论体系在新的历史条件下进一步丰富和发展了毛泽东思想。
\end{enumerate}
\end{enumerate}

\section{马克思主义中国化时代化的三次飞跃}

毛泽东思想、中国特色社会主义理论体系、习近平新时代中国特色社会主义思想。

\section{毛泽东思想形成发展的历史条件}

\begin{enumerate}[label=(\arabic*)]
\item 社会历史条件(国内背景):近代中国社会的特殊国情和中国革命的特殊性。

\item 时代背景(国际背景):19世纪末20世纪初,世界进入帝国主义和无产阶级革命时代,战争与革命成为时代主题。

\item 实践基础:中国共产党领导人民进行革命和建设的成功实践是毛泽东思想形成和发展的实践基础。
\end{enumerate}

\section{毛泽东思想活的灵魂(重要)}
它们有三个基本方面,即实事求是、群众路线、独立自主。实事求是是毛泽东思想的精髓。

\section{实事求是(重点)}

\begin{enumerate}[label=(\arabic*)]
\item 含义:
\begin{enumerate}[label=(\roman*)]
\item 实事求是,就是一切从实际出发,理论联系实际,坚持在实践中检验真理和发展真理。

\item "实事"就是客观存在着的一切事物,"是"就是客观事物的内部联系,即规律性,"求"就是我们去研究。
\end{enumerate}

\item 重要性:实事求是,是马克思主义的根本观点,是中国共产党人认识世界、改造世界的根本要求,是我们党的基本思想方法、工作方法、领导方法。
\end{enumerate}

\section{群众路线(重点)}

\begin{enumerate}[label=(\arabic*)]
\item 含义:
\begin{enumerate}[label=(\roman*)]
\item 群众路线,就是一切为了群众,一切依靠群众,从群众中来,到群众中去,把党的正确主张变为群众的自觉行动。

\item 坚持群众路线,就是坚持全心全意为人民服务的根本宗旨。全心全意为人民服务,是我们党一切行动的根本出发点和落脚点,是我们党区别于其他一切政党的根本标志。
\end{enumerate}

\item 重要性:
\begin{enumerate}[label=(\roman*)]
\item 群众路线都是我们党的生命线和根本工作路线,是我们党永葆青春活力和战斗力的重要传家宝。

\item 群众路线本质上体现的是马克思主义关于人民群众是历史创造者这一基本原理。
\end{enumerate}
\end{enumerate}

\section{独立自主}

\begin{enumerate}[label=(\arabic*)]
\item 含义:独立自主,就是坚持独立思考,走自己的路,就是坚定不移地维护民族独立、捍卫国家主权,把立足点放在依靠自己力量的基础上,同时积极争取外援,开展国际经济文化交流,学习外国一切对我们有益的先进事物。

\item 重要性:独立自主是中华民族的优良传统,是中国共产党、中华人民共和国立党立国的重要原则,是我们党从中国实际出发、依靠党和人民力量进行革命、建设、改革的必然结论、不论过去、现在和将来,我们都要把国家和民族的发展放在自己力量的基本点上,增强民族自尊心和自信心,坚定不移走自己的路。
\end{enumerate}

\section{毛泽东思想的历史地位}

\begin{enumerate}[label=(\arabic*)]
\item 马克思主义中国化时代化的第一个重大理论成果;

\item 中国革命和建设的科学指南;

\item 中国共产党和中国人民宝贵的精神财富。
\end{enumerate}

\section{新民主主义革命总路线的内容(重点)}

无产阶级领导的,人民大众的,反对帝国主义、封建主义和官僚资本主义的革命。

\section{新民主主义革命的对象}

\begin{enumerate}[label=(\arabic*)]
\item 分清敌友,这是革命的首要问题。近代中国社会的性质和主要矛盾,决定了中国革命的主要敌人就是帝国主义、封建主义和官僚资本主义。

\item 帝国主义是中国革命的首要对象。

\item 封建地主阶级是帝国主义统治中国和封建军阀实行专制统治的社会基础。是中国经济现代化和政治民主化的主要的、直接的障碍。

\item 官僚资本主义是依靠帝国主义、勾结封建势力、利用国家政权力量而发展起来的买办的、封建的国家垄断资本主义。
\end{enumerate}

\section{新民主主义革命的动力}

\begin{enumerate}[label=(\arabic*)]
\item 新民主主义革命的动力包括无产阶级、农民阶级、城市小资产阶级和民族资产阶级。

\item 无产阶级是中国革命最基本的动力。无产阶级是新的社会生产力的代表,是近代中国最进步的阶级,是中国革命的领导力量。

\item 农民是中国革命的主力军。他们深受帝国主义、封建主义和官僚资本主义的压迫和剥削,具有强烈的反帝反封建的革命要求。

\item 城市小资产阶级是无产阶级的可靠同盟者。城市小资产阶级,包括广大的知识分子、小商人、手工业者和自由职业者,同样受帝国主义、封建主义和官僚资本主义的压迫和掠夺。

\item 民族资产阶级也是中国革命的动力之一。半殖民半封建社会的民族资产阶级是一个带有两面性的阶级。它既不可能充当革命的主要力量,更不可能是革命的领导力量。
\end{enumerate}

\section{新民主主义的政治纲领}

推翻帝国主义和封建主义的统治,建立一个无产阶级领导的、以工农联盟为基础的、各革命阶级联合专政的新民主主义的共和国。

\section{新民主主义的经济纲领(重点)}

没收封建地主阶级的土地归农民所有,没收官僚资产阶级的垄断资本归新民主主义的国家所有,保护民族工商业。

\section{新民主主义的文化纲领}

新民主主义的文化,就是无产阶级领导的人民大众的反帝反封建的文化,即民族的科学的大众的文化。

\section{新民主主义革命道路的内容(重点)}

\begin{enumerate}[label=(\arabic*)]
\item 中国革命走农村包围城市、武装夺取政权的道路,根本在于处理好土地革命、武装斗争、农村革命根据地建设三者之间的关系。

\item 土地革命是中国革命的基本内容;武装斗争是中国革命的主要形式,是农村革命根据地建设和土地革命的强有力保证;农村革命根据地是中国革命的战略阵地,是进行武装斗争和开展土地革命的依托。
\end{enumerate}

\section{新民主主义革命的三大法宝(重要)}

\begin{enumerate}[label=(\arabic*)]
\item 统一战线、武装斗争、党的建设是党在中国革命中战胜敌人的三个主要法宝。

\item 统一战线是无产阶级政党策略思想的重要内容。

\item 武装斗争是中国革命的特点和优点之一。

\item 中国共产党要领导革命取得胜利,必须不断加强党的思想建设、组织建设和作风建设。

\item 统一战线和武装斗争是中国革命的两个基本特点,是战胜敌人的两个基本武器。统一战线是实行武装斗争的统一战线,武装斗争是统一战线的中心支柱,党的组织则是掌握统一战线和武装斗争这两个武器以实行对敌冲锋陷阵的英勇战士。
\end{enumerate}

\section{新民主主义社会中国的五种经济成分}

新民主主义社会不是一个独立的社会形态,而是由新民主主义向社会主义转变的过渡性社会形态。在新民主主义社会中,存在着五种经济成分,即社会主义性质的国营经济、半社会主义性质的合作社经济、农民和手工业者的个体经济、私人资本主义经济和国家资本主义经济。

\section{过渡时期的总路线}

从中华人民共和国成立,到社会主义改造基本完成,这是一个过渡时期。党在这个过渡时期的总路线和总任务,是要在一个相当长的时期内,逐步实现国家的社会主义工业化,并逐步实现国家对农业、对手工业和对资本主义工商业的社会主义改造。

\section{党在过渡时期的主要内容}

"一化三改"。"一化"即社会主义工业化,"三改"即对个体农业、手工业和资本主义工商业的社会主义改造。他么之间相互联系,不可分离,可以比喻为鸟的"主体"和"两翼"。其中,"一化"是"主体","三改"是"两翼"。

\section{社会主义改造的历史经验}

\begin{enumerate}[label=(\arabic*)]
\item 坚持社会主义工业化建设与社会主义改造同时并举。

\item 采取积极引导、逐步过渡的方式。

\item 用和平方式进行改造。
\end{enumerate}

\section{《论十大关系》的基本方针}

努力把党内外、国内外的一切积极的因素,直接的、间接的积极因素,全部调动起来。

\section{社会主义社会的矛盾}

\begin{enumerate}[label=(\arabic*)]
\item 关于社会主义社会的基本矛盾。

\item 关于我国社会的主要矛盾和根本任务。

\item 关于社会主义社会两类不同性质的矛盾。

\item 关于正确处理两类不同性子社会矛盾的基本方法。

\item 关于正确处理人民内部矛盾的方针。

\item 关于严格区分两类不同性质矛盾和正确处理人民内部矛盾的目的和意义。
\end{enumerate}

\section{中国工业化道路的总方针}

毛泽东提出了以农业为基础,以工业为主导,以弄轻重为序发展国民经济的总方针,以及以整套"两条腿走路"的工业化发展思路,即重工业和轻工业同时并举,中央工业和地方工业同时并举,沿海工业和内地工业同时并举,大型企业和中小型企业同时并举,等等。

\section{初步探索的意义}

\begin{enumerate}[label=(\arabic*)]
\item 巩固和发展了我国的社会主义制度。

\item 为开创中国特色社会主义提供了宝贵经验、理论准备、物质基础。

\item 丰富了科学社会主义的理论和实践。
\end{enumerate}

\section{初步探索的经验与教训}

\begin{enumerate}[label=(\arabic*)]
\item 必须把马克思主义与中国实际相结合,探索符合中国特点的社会主义建设道路。
\item 必须正确认识社会主义社会的主要矛盾和根本任务,集中力量发展生产力。
\item 必须从实际出发进行社会主义建设,建设规模和速度要与国力相适应,不能急于求成。
\item 必须发展社会主义民主,健全社会主义法制。
\item 必须坚持党的民主集中制和集体领导制度,加强执政党建设。
\item 必须坚持对外开放,借鉴和吸收人类文明成果建设社会主义,不能关起门来搞建设。
\end{enumerate}

\section{中国特色社会主义理论体系形成发展过程(重点)}

\begin{enumerate}[label=(\arabic*)]
\item 邓小平理论标志着中国特色社会主义理论体系的形成。

\item "三个代表重要思想"是中国特色社会主义理论体系的跨世纪发展。

\item 科学发展观是中国特色社会主义理论体系在新世纪新阶段的新发展。

\item 习近平新时代中国特色社会主义是中国特色社会主义理论体系在新时代的新篇章。
\end{enumerate}

\section{社会主义的本质}

\begin{enumerate}[label=(\arabic*)]
\item 社会主义的本质,是解放生产力,消灭剥削,消除两极分化,最终达到共同富裕。

\item 邓小平对社会主义本质的概括,既包括了社会主义社会的生产力问题,又包括了社会主义社会的生产关系问题,是一个有机的整体。

\begin{enumerate}[label=(\roman*)]
\item 首先,它突出强调解放和发展生产力在社会主义社会发展中的重要地位,纠正了过去关于发展生产力的一些错误观念,反映了中国社会主义整个历史阶段尤其是初级阶段特别需要注重生产力发展的迫切要求,并明确了社会主义基本制度建立后还要通过改革进一步解放生产力。

\item 其次,它突出地强调"消灭剥削,消除两极分化,最终达到共同富裕",从生产关系和发展目标角度认识和把握社会主义本质。
\end{enumerate}
\end{enumerate}

\section{邓小平理论的精髓(重点)}

\begin{enumerate}[label=(\arabic*)]
\item 解放思想、实事求是是邓小平理论的精髓。

\item 解放思想、实事求是贯穿邓小平理论形成发展的全过程。邓小平深刻阐明了解放思想和实事求是的辩证统一关系,只有解放思想才能达到实事求是,只有实事求是才是真正的解放思想。
\end{enumerate}

\section{社会主义初级阶段}

\begin{enumerate}[label=(\arabic*)]
\item 党的十三大召开前夕,邓小平指出:"我们党的十三大要阐述中国社会主义是处在一个什么阶段,就是处在初级阶段,是初级阶段的社会主义。社会主义本身是共产主义的初级阶段,而我们中国又处在社会主义的初级阶段,就是不发达的阶段。一切都要从这个实际出发,根据这个实际来制订规划。"

\item 党的十三大系统地阐述了社会主义初级阶段的科学内涵。首先,阐明了社会主义初级阶段这个论断包括两层含义:

\begin{enumerate}[label=(\roman*)]
\item 第一,我国社会已经是社会主义社会。我们必须坚持而不能离开社会主义。

\item 第二,我国的社会主义社会还处在初级阶段。我们必须从这个实际出发,而不能超越这个阶段。
\end{enumerate}

\item 其次,强调了社会主义初级阶段的长期性。

\item 最后,阐述了社会主义初级阶段的基本特征。
\end{enumerate}

\section{社会主义改革开放理论(为什么改革是中国第二次革命?)(重要)}

\begin{enumerate}[label=(\arabic*)]
\item 性质:
\begin{enumerate}[label=(\roman*)]
\item 改革是一场深刻的社会变革,是中国的第二次革命,是实现中国现代化的必由之路。

\item 改革不是原有经济体制的细枝末节的修补,它的实质和目标是要从根本上改变束缚我国生产力发展的经济体制,建立充满生机和活力的社会主义新经济体制,同时相应地改革政治体制和其他方面的体制,以实现中国的社会主义现代化。

\item 改革不是一个阶级推翻另一个阶级那种原来意义上的革命,不是也不允许否定和抛弃我们建立起来的社会主义基本制度,而是社会主义制度的自我完善和发展。
\end{enumerate}

\item 改革是社会主义社会发展的直接动力。改革是从根本上改变束缚生产力发展的经济体制,促进生产力的发展,解决社会主义社会发展的动力问题。

\item 改革是一项崭新的事业,是一个大试验。不能因循守旧,四平八稳,不能不顾条件,急于求成。判断改革和各方面是非得失,归根到底,主要看是否符合"三个有利于"标准。
\end{enumerate}

\section{“三个代表”重要思想的核心观点(重点)}

\begin{enumerate}[label=(\arabic*)]
\item 始终代表中国先进生产力的发展要求,就是党的理论、路线、纲领、方针、政策和各项工作,体现不断推动社会生产力的解放和发展的要求,通过发展生产力不断提高人民群众的生活水平。

\item 始终代表中国先进文化的前进方向,就是党的理论、路线、纲领、方针、政策和各项工作,必须努力体现发展面向现代化、面向世界、面向未来的,民族的科学的大众的社会主义文化的要求。为我国经济发展和社会进步提供精神动力和智力支持。

\item 始终代表中国最广大人民的根本利益,就是党的理论、路线、纲领、方针、政策和各项工作,必须坚持把人民的根本利益作为出发点和归宿,充分发挥人民群众的积极性主动性创造性,使人民群众不断获得切实的经济、政治、文化利益。
\end{enumerate}


\section{和平共处五项原则(重要)}

互相尊重主权和领土完整、互不侵犯、互不干涉内政、平等互利。

\section{科学发展观的科学内涵是什么?}
\begin{enumerate}[label=(\arabic*)]
\item 推动经济社会发展是科学发展观的第一要义;
\item 以人为本是科学发展观的核心立场;
\item 全面协调可持续是科学发展观的基本要求;
\item 统筹兼顾是科学发展观的根本方法.
\end{enumerate}

\section{构建社会主义和谐社会的总要求?(重点)}

民主法治,公平正义,诚信友爱,充满活力,安定有序,人与自然和谐相处,是构建社会主义和谐社会的总要求。






\chapter{习概期末大题}

\section{习近平新时代中国特色社会主义思想主要内容}

党的十九大、十九届六中全会提出的“十个明确”“十四个坚持”“十三个方面成就”。

\section{习近平新时代中国特色社会主义思想的世界观和方法论}

"六个必须坚持",就是必须坚持人民至上,自信自立,守正创新,问题导向,系统观念,胸怀天下。

\section{两个确立}

党确立习近平同志党中央的核心、全党的核心地位.确立习近平新时代中国特色社会主义思想的指导地位.


\section{四个自信}

道路自信,制度自信,理论自信,文化自信.


\section{四个全面具体内容}

全面建设社会主义现代化国家,全面深化改革,全面依法治国,全面从严治党.

\section{全面建成社会主义现代化强国的战略安排}

分两步走∶从2020年到2035年基本实现社会主义现代化,从2035年到本世纪中叶把我国建成富强民主文明和谐美丽的社会主义现代化强国.

\section{中国共产党党的初心与使命}

我们党始终坚守着为中国人民谋幸福,为中华民族谋复兴的初心和使命.

\section{改善民生从哪几方面入手}

完善分配制度,实施就业优先战略,健全社会保障体系,推进健康中国建设.

\section{全民推进国防和军队现代化的战略安排是}

\begin{enumerate}
\item 到2022实现建军一百年奋斗目标;

\item 到2035基本实现国防和军队现代化;

\item 到本世纪中叶把人民军队全面建成世界一流军队.
\end{enumerate}

\section{中国式现代化的中国特色}

中国式现代化是人口规模巨大的现代化,是全体人民共同富裕的现代化,是物质文明与精神文明相协调的现代化,是人与自然和谐共生的现代化,是走和平发展道路的现代化。

\section{中国特色社会主义政治制度体系有什么构成}
\begin{enumerate}
\item 中国特色社会主义政治制度是由根本制度,基本制度,重要制度构成;

\item 根本制度是人民代表大会制度;

\item 基本制度是中国共产党领导的多党合作和政治协商制度、民族区域自治制度和基层群众自制制度。
\end{enumerate}

\section{为中华民族谋复兴的初心和使命}

为中国人民谋幸福,为中华民族谋复兴。

\section{新发展理念}

创新,协调,绿色,开放,共享。

\section{社会主义基本经济制度}

公有制为主体、多种所有制经济共同发展,按劳分配为主体、多种分配方式并存,社会主义市场经济制体制等.

\section{社会主义基本经济制度是什么}

\begin{enumerate}
\item 毫不动摇巩固和发展公有制经济,毫不动摇鼓励支持引导非公有制经济发展。

\item 公有制经济包括国有经济和集体经济,还包括混合所有制经济中的国有成分和集体成分。

\item 非公有制经济包括个体经济,私营经济,港澳台投资经济,外商投资及混合所有制经济中的非国有成分和非集体成分。
\end{enumerate}

\section{如何理解人民对美好生活的向往就是党的奋斗目标?}

\begin{enumerate}
\item 坚持人民至上,必须始终把人民放在心中最高的位置,想人民之所想,行人民之所嘱,必须人民对美好生活的向往作为党的奋斗目标,团结带领人民共同创造美好的未来。

\item 为人民谋幸福是党始终坚守的初心,让人民过上好日子是党一贯的追求。为了让人民过上好日子,我们党团结带领人民实现民族独立、人民解放,建立了新中国;我们党团结带领人民建立社会主义制度、改变了一穷二白的国家面貌;我们党团结带领人民实行改革开放,解放和发展社会生产力,解决温饱、实现总体小康、全面建设小康社会;我们党团结带领人民胜全面建成小康社会,打赢脱贫攻坚战,迈上全面建设社会主义现代化国家新征程,人民对美好生活的向往不断变成现实。我们党的百年奋斗,归根到底就是让中国人民过上好日子,实现国家富强民族复兴。

\item 人民对美好生活的向往就是党的奋斗目标,这体现了我国社会主要矛盾转化对党和国家工作的新要求。进入新时代,我国社会主要矛盾发生了转化,人民的美好生活需要呈现出多样化多层次多方面的特点。对幼有所学、学有所教、劳有所得、病有所医、老有所养、住有所居,弱有所扶有了更高的期盼,对民主、法治、公平、正义、安全、环境等提出了更高的要求,人民的美好生活需要有了更加丰富更加深刻的时代内涵。必须牢牢把握我国社会主要矛盾的新变化,适应人民对高品质生活的新期待,在高质量发展中努力为人民创造更美好、更幸福的生活。
\end{enumerate}


\section{如何理解“习近平指出时代是出卷人,我们是答卷人,人民是阅卷人。”这句话}

\begin{enumerate}
\item 我们党是全心全意为人民服务的党,党的执政水平成效不是由自己说了算,必须而且只能由人民来评判。

\item 让群众满意是我们党做好一切工作的价值取向和根本标准。人民群众的意见是最好的尺子,最能衡量我们工作的长短优劣。生活过得好不好,人民最有发言权。什么是好事实事,要看群众感受,由群众来评判。要关注民情、顺应民意,人民群众赞成什么、期盼什么,就要坚持和推动什么;人民群众反对什么、痛恨什么,就要防范和纠正什么。对群众的呼声,要做到事事有着落渐渐有回应,用情用心为群众排忧解难,千方百计增进人民福祉。

\item 必须牢固树立和践行正确的政绩观。坚持把好事实是做到群众的心坎上,真正做到对历史和人民负责,这是党员干部应当始终坚守的态度。要以功成不必在我”的精神境界和工程必定由我的历史担当,多做为后人做铺垫、打基础、立长远的好事,追求人民群众的好口碑、历史沉淀之后的真评价。
\end{enumerate}

\section{如何理解绿色青山就是金山银山内在联系?}


辩证统一,相辅相成.
\begin{enumerate}
\item 发展经济不能对资源和生态环境竭泽而渔,生态环境保护也不是舍弃经济发展而缘木求鱼,而是要坚持在发展中保护、在保护中发展

\item 既要绿水青山也要金山银山。宁要绿水青山,不要金山银山,而且绿水青山就是金山银山。

\item 良好生态环境本身蕴涵着无穷的经济价值,能够源源不断创造综合效益。一方面,绿水青山充分发挥经济社会效益,为社会发展提供材料,推动社会生产力的绿色发展,另一方面绿色发展将绵延绿水青山的生机,保护和改善自然生产力。因此绿水青山和金山银山是良性循环。
\end{enumerate}


\section{大学生如何建立美丽中国?}

坚持把建设美丽中国转化为全体人民自觉行动。生态文明是人民群众共同参与共同建设共同享有的事业,每个人都是生态环境的保护者、建设者、受益者。要在始终坚持用最严格制度最严格法制保护生态环境、保护常态化外部压力的同时,激发起全社会共同呵护生态环境的内生动力,增强全民节约意识、环保意识、生态意识,培育生态道德和行为准则,开展全民绿色行动,动员全社会都以实际行动减少能源资源消耗和污染排放,加快形成绿色生产方式和生活方式。


\section{如果理解"科教兴国战略,人才强国战略,创新驱动发展战略"?}

\begin{enumerate}
\item 三者既相互融合又各有侧重,要加强统筹,有效联动。

\item 科教兴国战略,就是要全面落实科学技术是第一生产力的思想,坚持教育优先发展,把科技和教育作为经济社会发展的重中之重,促进数育同经济、科技的密切结合,把经济建设转到依靠科技进步和提高劳动者素质的轨道上来,为实现经济发展提供科技支撑和人才保障。

\item 人才强国战略就是要牢固树立人才资源是第一资源的理念,把人才队伍建设提升到国家战略的高度,营造良好人才创新生态环境,充分发挥备类人才的积极性、主动性和创造性,开创人才辈出、人尽其才的新局面,把我国由人口大国转化为人才资源强国。

\item 创新驱动发展战略,就是要坚持创新是第一动力,把创新驱动落实到现代化建设整个进程和各个方面,以创新推动经济转型发展,全面提升创新能力和效率,把创新发展主动权牢牢掌握在自己手中。
\end{enumerate}


\section{保障和改善民生应重点从哪些方面入手?}

\begin{enumerate}
\item 完善分配制度.
收入分配是民生之源,是改善民生、实现发展成果由人民共享最重要最直接的方式。分配制度是促进共同富裕的基础性制度。在社会再生产过程中,分配是联结生产与消费的重要纽带,对激励生产者的积极性和创造性、对保证和促进消费,具有重要作用。要正确处理生产和分配的关系,既要把“蛋糕”做大做好,又要把“蛋糕”切好分好,使全体人民共享改革发展的成果。

\item 实施就业优先战略.
就业是最基本的民生,是劳动者赖以生存和发展的基础、共享经济发展成果的基本条件,关系到亿万劳动者及其家庭的切身利益。解决好就业问题,是民生改善的“温度计”,对国家长治久安具有重要支撑作用。实施就业优先战略,把就业摆在经济社会发展优先位置,突出就业作为基本民生的重要作用,有利于不断扩大就业容量、创造和增加收入、改善人民生活品质。

\item 健全社会保障体系.
社会保障体系是由国家立法规定并以国家作为给付义务主体,为保障社会成员的基本生活和福利、对生活困难社会成员给予物质或服务帮助的各项措施的统称,主要涉及社会保险、社会救助、社会福利、社会优抚等方面。健全和完善社会保障体系是新时代加强社会建设的重要着力点。坚持以增强公平性、适应流动性、保证可持续性为重点,推动我国社会保障体系建设进入快车道,基本建成以社会保险为主体、功能完备的社会保障
体系。

\item 推进健康中国建设.
人民健康是社会文明进步的基础,是民族昌盛和国家富强的重要标志,是促进人的全面发展的必然要求。卫生健康事业在经济社会发展中处于基础性地位,拥有健康的人民意味着拥有更强大的综合国力和可持续发展能力。加快提高卫生健康供给质量和服务水平,是适应我国社会主要矛盾变化、满足人民美好生活需要的要求,也是实现经济社会更高质量、更有效率、更加公平、更可持续、更为安全发展的基础。
\end{enumerate}


\section{知识点}

\begin{enumerate}
\item 十九大把习近平新时代中国特色社会主义思想写入党章。
\item “六个必须坚持”就是必须坚持人民至上,自信自立,守正创新,问题导向,系统观念,胸怀天下。
\item 实现中华民族伟大复兴的本质是国家富强,民族振兴,人民幸福。
\item 中国式现代化的中国特色:中国式现代化是人口规模巨大的现代化,是全体人民共同富裕的现代化,是物质文明与精神文明相协调的现代化,是人与自然和谐共生的现代化,是走和平发展道路的现代化。
\item 中国式现代化的本质要求:坚持中国共产党的领导,坚持中国特色社会主义,实现高质量发展,发展全过程人民民主,丰富人民精神世界,实现全体人民共同富裕,促进人与自然和谐共生,推动构建人类命运共同体,创造人类文明新形态。
\item 中国特色社会主义最本质特征:中国共产党领导。
\item 党的领导是全面的,系统的,整体的。
\item 坚持以人民为中心,是新时代坚持和发展中国特色社会主义的根本立场,是贯穿党治国理政全部活动的一条红线。
\item 习近平指出党开天辟地发生历史性变革的根本原因是我们党始终坚守了为中国人民谋幸福,为中华民族谋复兴的初心和使命。
\item 人民立场是中国共产党的根本政治立场,是我们党区别于其他政党的显著标志。
\item 全面深化改革的总目标:完善和发展中国特色社会主义制度,推进国家治理体系和治理能力现代化。
\item 坚定不移贯彻创新,协调,绿色,开放,共享的新发展理念。
\item 构建以国内大循环为主体,国内国际双循环相互促进的新发展格局。
\item 走中国特色社会主义政治发展道路,必须坚持党的领导,人民当家作主,依法治国有机统一。
\item 党的二十大指出:“全过程人民民主是社会主义民主政治的本质属性”,深刻彰显了我国人民民主的鲜明特色和显著优势,实现了过程民主和成果民主、程序民主和实质民主、直接民主和间接民主、人民民主和国家意志相结合。
\item 全过程人民民主是全链条、全方位、全覆盖的民主。
\item 全过程民主是最广泛,最真实,最管用的民主。
\item 建设中国特色社会主义法治体系,建设社会主义法治国家,是坚持和发展中国特色社会主义的内在要求。
\item 全面推进科学立法,严格执法,公正司法,全民守法。
\item 坚持马克思主义在意识形态领域指导地位的根本制度,是坚持和巩固我国社会主义制度、保证我国文化建设正确方向的必然要求。
\item 中国共产党的先驱们创建了中国共产党,形成了坚持真理,坚守理想,践行初心,担当使命,不怕牺牲,英勇斗争,对党忠诚,不负人民的伟大建党精神。
\item (背)经过接续奋斗,我国社会主义建设全面加强,幼有所育,学有所教,劳有所得,病有所医,老有所养,住有所居,弱有所扶得到更好实现,人民安居乐业,社会安定有序的良好局面不断巩固发展,人民对美好生活的向往不断变为现实。
\item 人的命脉在田,田的命脉在水,水的命脉在山,山的命脉在土,土的命脉在林和草,这个生命共同体是人类生存发展的物质基础。
\item 2014年4月15日,总体国家安全观。
\item 总体国家安全观的关键是“总体”。强调“大安全”理念,涵盖政治、军事、国土、经济、金融、文化、社会、科技、网络、粮食、生态、资源、核、海外利益、太空、深海、极地、生物、人工智能、数据等诸多领域。
\item 把维护政治安全放在首位。
\item 党的十九大明确指出,党在新时代的强军目标是建设一支听党指挥,能打胜仗,作风优良的人民军队,把人民军队建设成为世界一流军队。
\item “一国两制”是针对台湾。
\item 中国在坚持和平共处五项原则基础上同各国发展友好合作,深化拓展平等,开放,合作的全球伙伴关系。
\item 党的自我革命是跳出历史周期率的第二个答案。
\end{enumerate}










\end{document}